% intro.tex
\section*{Introduction}

Jeremiah is unusual in that in the time of Jesus and the apostles it existed in two distinct forms with essentially the same message but with different ordering and emphases. The version that comes to us in our English bibles is based on the longer Hebrew Masoretic Text (MT) version, but the version quoted by the apostles and prophets of the New Testament is the shorter Greek Septuagint (LXX) version which is in turn based on a corresponding Hebrew version that was known in the first century but unknown in modern times until its rediscovery in the Dead Sea Scrolls. Scholars think the Septuagint version represents the older textual tradition that has been expanded on in the MT, but almost all modern editions translate from the MT, although they may include notes as to the differences.

\hspace{1.5em} This document presents Jeremiah in LXX order using the 1870 Brenton English translation obtained from \url{https://ebible.org/Scriptures/eng-Brenton_usfm.zip}
in parallel with an MT-based English translation, the World English Bible (WEB) obtained from \url{https://ebible.org/Scriptures/eng-web_usfm.zip}. These translations were selected in preference to more contemporary versions because they are in the public domain, meaning this document can also be made freely available and free to copy. The Brenton translation has been lightly edited to bring up to date some of the more archaic usages. The spelling of some more promenent names has also been aligned with more usual English usage (Jeremiah rather than Jeremias, Judah rather than Juda, etc.) This updating sometimes introduces ambiguity between singular and plural you which the use of the archaic 'thee' and 'ye' distinguishes. Where it was possible that this ambiguity has been introduced, plural 'you' is marked with a double dagger: you\textsuperscript{‡}. The kind of edits made is illustrated below.

\begin{center}
  \begin{tabular}{l l}
  \hline
  Where Brenton has&Edited to\\
  \hline
  thee&you\\
  thou camest&you came\\
  I know not; be not afraid&I don't know; don't be afraid\\
  whomsoever&whoever\\
  What seest thou?; Thou hast well seen&What do you see?; You have seen well\\
  do thou gird up thy loins&gird up your loins\\
  a brazen wall&a bronze wall\\
    Hear ye the word&Hear the word\\
    peradventure&perhaps\\
    hence; thence; hither; thither&from here; from there; here; there\\
\hline
\end{tabular}
\end{center}
Words in italics indicating translator's addition of words absent in the original but necessary for the English to make sense are all as in Brenton. Text set in small caps and descriptive headings (such as in 14:1) are also as specified by Brenton.

The WEB text has not been edited apart from the re-ordering necessary to present parallel passages together. No verses have been divided, so you can determine the exact position of the WEB text in its original order by referring to the verse references. Footnotes reproduce 
notes from the WEB source text.

 Verse numbering follows
the Brenton LXX (including a/b suffixes where present). Verse matching between versions was prepared based on the information provided at \url{https://www.ccel.org/bible/brenton/Jeremiah/appendix.html}. Where content appears in only one edition, the other column has a verse number with empty content.

\bigskip
\noindent Glen Prideaux\\
February 2026

\bigskip
\noindent\textbf{Left:} LXX English (Brenton)\\
\textbf{Right:} MT English (WEB)